\documentclass[a4paper,12pt]{article}

\usepackage[T2A]{fontenc}
\usepackage[utf8]{inputenc}
\usepackage[russian]{babel}
\usepackage{amsmath,amssymb,amsthm}
\usepackage{mathtools}
\usepackage{booktabs,array}
\usepackage{forest}
\usepackage[hidelinks]{hyperref}
\usepackage{tikz}
\usetikzlibrary{arrows.meta,positioning}

\title{Лабораторная работа №1}
\author{Чистяков Илья ИУ9-52Б}

\begin{document}
\maketitle

\section*{Изначальная система переписывания}

Рассматривается система переписывания строк (вариант 24), где алфавит
\[
\Sigma = \{a,b,c\},
\]
а множество правил задано следующим образом:
\[
\begin{aligned}
\texttt{ aa} &\to \varepsilon\\
\texttt{ bb} &\to \texttt{cccc}\\
\texttt{ cc} &\to \texttt{acb}\\
\texttt{  abc} &\to \texttt{aabbcc}\\
\texttt{ baabaac} &\to \texttt{cbba}
\end{aligned}
\]


\section*{Завершимость}

Рассмотрим количество применений каждого из правил для произвольной строки. Количетво применений 4 и 5 правила конечно, так как система не может сама породить ни \texttt{abc} ни \texttt{baabaa}. Следовательно для их применения нужно использовать буквы из начальной строки, а их количество конечно. Количество применений первого и второго правила также конечно, так как строки \texttt{aa} и \texttt{bb} можно бескнечно порождать только четвертым и пятым правилом, количество применений которых, как мы выяснили, конечно. Отсюда количество применений третьего правила также конечно и вся система завершима.


\section*{Конечность}


Возьмём семейство слов  $w_n = (ac)^n .$ Каждое ($w_n$) принадлежит отдельному классу эквивалентности.



\section*{Конфлюэнтность}

Изначальная система не является конфлюэнтной ни локально, ни глобально. Это можно увидеть на примере строки 
\texttt{ccc}:

\begin{center}
\begin{forest}
for tree={
  draw,
  rounded corners,
  align=center,
  s sep=10mm,
  l sep=12mm,
  edge={-stealth, thick}, 
}
[ccc
  [cacb]
  [acbc]
]
\end{forest}
\end{center}

Дополним систему по алгоритму Кнута-Бендикса. Сначала введём лексикографический порядок в систему и переупорядочим правила:

\[
\begin{aligned}
\texttt{ aa} &\to \varepsilon\\
\texttt{ cccc} &\to \texttt{bb}\\
\texttt{ acb} &\to \texttt{cc}\\
\texttt{  aabbcc} &\to \texttt{abc}\\
\texttt{ baabaac} &\to \texttt{cbba}\\
\end{aligned}
\]

Её можно упростить до


\[
\begin{aligned}
\texttt{ aa} &\to \varepsilon\\
\texttt{ cccc} &\to \texttt{bb}\\
\texttt{ acb} &\to \texttt{cc}\\
\texttt{  bbcc} &\to \texttt{abc}\\
\texttt{ cbba} &\to \texttt{bbc}\\
\end{aligned}
\]
Пополняя по кнуту бендиксу получаем

\[
\begin{aligned}
\texttt{ aa}   &\to \varepsilon\\
\texttt{ acb}  &\to \texttt{cc}\\
\texttt{ acc}  &\to \texttt{cb}\\
\texttt{ cbb}  &\to \texttt{bbb}\\
\texttt{ bbbb} &\to \texttt{bb}\\
\texttt{ abb}  &\to \texttt{bb}\\
\texttt{ ccc}  &\to \texttt{bbb}\\
\texttt{ ccb}  &\to \texttt{bbb}\\
\texttt{ bc}   &\to \texttt{bb}\\
\texttt{ bba}  &\to \texttt{bb}\\
\end{aligned}
\]


\section*{Тестирование}
\subsection*{Фаззинг}


В фаззинг тестировании, сгенерированное случайное слово я приводил сначало к нормальной форме в $\tau$. А после этого приводил эту нормальную форму и изначальное слово к нормальной форме в $\tau'$. Эквивалентность этих двух нормальных форм означает эквивалентность двух SRS. Несмотря на то, что цепочки переписывания в моей реализации получаются детерминированными, основную свою функцию тестирование выполняет.



\subsection*{Метаморфное тестирование}

В этом тестировании реализовал недетерминированные цепочки переписывания. В качестве инвариантов были выбраны две характеристики:

1. Четность количества b + c в слове 

2. Длина слова




\end{document}